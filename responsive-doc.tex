\documentclass{ltxdoc}

\usepackage[noautomatic]{responsive}
\title{The Responsive Package}
\author{Michal Hoftich\thanks{\url{michal.h21@gmail.com}}}
\NewDocElement[macrolike = false ,
toplevel = false,
idxtype
= key,
idxgroup = Package keys,
% printtype = \textit{key}
]{Key}{key}

\makeindex

\begin{document}
\maketitle
\begin{abstract}
Responsive design aims to display text and other design elements well on
variety of outputs, including electronic devices or various paper sizes.
It originated on the Web, where it uses Cascading Style Sheets to change 
design elements. 

This package tries to achieve similar result with \LaTeX. The main aim is to 
support output of multiple editions of one document with 
readable font sizes regardless of the physical size of the produced PDF,
from smartphones or e-ink readers to printed A4.
\end{abstract}
\tableofcontents

\section{Features}

\subsection{Fitting Font Size to the Page Size}

\subsection{Typographic Scale}

\subsection{Media Queries}

\section{Package Options}

The following options can be passed as package uptions used in \cmd{\usepackage}, 
or later in the document using the \cmd{\ResponsiveSetup} command.

\medskip

\noindent \DescribeKey{characters} Set aproximate number of characters that should fit to a text line.

\noindent \DescribeKey{noautomatic}  Disable automatic setting of font sizes in the document. You can set them 
later using the \cmd{\setsizes} command.

\noindent \DescribeKey{scale} Select typographic scale




\section{Commands}

\DescribeMacro\setsizes
\cmd\setsizes\oarg{xxx}\marg{characters per lines}

% \PrintIndex

\section{Examples}

\end{document}
