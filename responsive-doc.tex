\documentclass{ltxdoc}

\usepackage[noautomatic]{responsive}
\usepackage{linebreaker}
\usepackage{fontspec}
\usepackage{caption}
\usepackage{subcaption}
\usepackage{graphicx}
\usepackage{microtype}
\setmainfont{EB Garamond}
\usepackage{lipsum}
\title{The Responsive Package}
\author{Michal Hoftich\thanks{\url{michal.h21@gmail.com}}}
\NewDocElement[macrolike = false ,
toplevel = false,
idxtype
= key,
idxgroup = Package keys,
% printtype = \textit{key}
]{Key}{key}

\newcommand\printsize[1]{\csname #1\endcsname\par\noindent Sample\par}
\newcommand\showscale[2][.5\textwidth]{%
  \noindent\fbox{%
    \begin{minipage}{#1}
      \ResponsiveSetup{#2}
      \setsizes[34]{25}
      \printsize{huge}
      \printsize{LARGE}
      \printsize{Large}
      \printsize{large}
      \printsize{normalsize}
      \printsize{small}
      \printsize{footnotesize}
    \end{minipage}
  }
    \ifx\relax#2\relax Default parameters\else\texttt{#2}\fi\hfill\null
}

\makeindex

\begin{document}
\maketitle
\begin{abstract}
Responsive design aims to display text and other design elements well on
variety of outputs, including electronic devices or various paper sizes.
It originated on the Web, where it uses Cascading Style Sheets to change 
design elements. 

This package tries to achieve similar result with \LaTeX. The main aim is to 
support output of multiple editions of one document with 
readable font sizes regardless of the physical size of the produced PDF,
from smartphones or e-ink readers to printed A4.
\end{abstract}
\tableofcontents

\section{Features}

\begin{figure}[htp]
  \null\hfill\begin{minipage}{150pt}

    % \ResponsiveSetup{#2g}
    \setsizes[32]{56}

    
    {\Huge\par\noindent Huge text\par}

    \lipsum[1]
  \end{minipage}\hfill\null
  \caption{Example of responsive typography on a smaller page}
\end{figure}

\subsection{Fitting Font Size to the Page Size}

\subsection{Typographic Scale}

A typographic scale is a system of organizing type sizes in a harmonious manner, like on musical scale. 



\showscale{}
\showscale{scale=heptatonic}
\showscale{scale=tritonic}
\showscale{scale=golden}
\showscale{scale=none,ratio=1.3,number=2}

\subsection{Media Queries}

\newcommand\mdtest{%
  \mediaquery{max-textwidth=4cm}
  {\ResponsiveSetup{lineratio=38}\setsizes{45}}
  {\ResponsiveSetup{lineratio=34}\setsizes{60}}
  \lipsum[1]
}

\begin{figure}
  % \centering
  % \begin{subcaptionblock}{0.25\textwidth}
  \begin{subcaptiongroup}
    \begin{minipage}{3.9cm}
      { \mdtest
        
      }
    \caption{textwidth=3.9cm}
    \end{minipage}
    \hspace{2em}
    \begin{minipage}{6cm}
      { \mdtest 

      }
      \caption{\texttt{\textbackslash textwidth}=6cm}
    \end{minipage}
  \end{subcaptiongroup}
  % \end{subcaptionblock}
  \caption{Example of media query usage.}
\end{figure}



\section{Package Options}

The following options can be passed as package uptions used in \cmd{\usepackage}, 
or later in the document using the \cmd{\ResponsiveSetup} command.

\medskip

\noindent \DescribeKey{characters} Set aproximate number of characters that should fit to a text line.

\noindent \DescribeKey{noautomatic}  Disable automatic setting of font sizes in the document. You can set them 
later using the \cmd{\setsizes} command.

\noindent \DescribeKey{scale} Select typographic scale

\noindent \DescribeKey{number} Number of steps before the scale reaches the next multiplication point.

\noindent \DescribeKey{ratio} 

\noindent \DescribeKey{lineratio} 

\noindent \DescribeKey{boxwidth} Width of line that is used for font size calculations in the \cmd{\setsizes} command.




\section{Commands}

\DescribeMacro\setsizes
\cmd\setsizes\oarg{xxx}\marg{characters per lines}

% \PrintIndex

\section{Troubleshooting}

Occasionally, you can run into issues caused by the change of font sizes. \LaTeX\ sets lot of parameters 
depending on the base font size. We try to recalculate them according to the new font size, but you can still
experience some issues, described in following subsections.

\subsection{Error Messages From the Output Routine }

this trick should fix vbox errors in the output routine. 
we calculate exact box for the font size, but often,
we would get error when a bigger font is used on the page,
for example in chapters

There are several 
  % source: https://tex.stackexchange.com/a/62318/2891
  \def\@textbottom{\vskip \z@ \@plus \resp_font_size \@minus \resp_font_size}
\section{Examples}

\end{document}
